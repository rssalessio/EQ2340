\section{Forward Algorithm and logprob function}
The verification of the forward algorithm and the logprob function will be explained in the following section.\\
The first part describes the method that has been applied to correct the code as well as the deployed tests. Thereafter, the code itself is analyzed in more detail followed by a list of errors. This list will not only contain the error source, but also what consequences the erros could have and how severe they are. Then it is explained, how these issues are fixed. In the last part of this section, the logprob function is depicted.

\subsection{Procedure}
The general verification approach has been a peer review approach. One person started correcting the code up until the expected results have been obtained. While verifying, the wrong code segments have only been uncommented and additional code has been added. Using this procedure, it is easy for the peer reviewer to see the differences between old and new code. The second person (peer reviewer) reviewed the code once again having a special look at the differences of the original and the corrected version.\\
For the verification procedure itself, the algorithm stated in the course compendium has been compared with the matlab code by pure looking. Additionally, the matlab debugger has been used in order to check dimensions of variables and in order to find syntax or runtime errors.\\
After the two separated work sessions, a discussion session has been held in order to finalize the code and run the tests. The following test has been used for the finite HMM
\begin{lstlisting}
%% finite test case
A       = [0.9 0.1 0;0 0.9 0.1];
q       = [1 0];
mc      = MarkovChain(q,A);

B(1)    = GaussD('Mean',0,'StDev',1);
B(2)    = GaussD('Mean',3,'StDev',2);

x   = [-0.2 2.6 1.3];
pX = prob(B,x);
hMM = HMM(mc, B);

[alfaHat, c]=forward(mc,pX)
logprob(hMM,x)
\end{lstlisting}
and for the infinite test case
\begin{lstlisting}
%% infinite test case
clc; clear all; close all;

A       = [0.3 0.7 0;0 0.5 0.5; 0 0 0.1];
q       = [1;0;0];
mc      = MarkovChain(q,A);

B(1)    = DiscreteD([1 0 0 0]);
B(2)    = DiscreteD([0 0.5 0.4 0.1]);
B(3)    = DiscreteD([0.1 0.1 0.2 0.6]);

x   = [1 2 4 4 1];

pX  = prob(B,x);

[alfaHat, c]=forward(mc,pX)
\end{lstlisting}

The correct values of these cases can be found in the course compendium - for the finite case
\begin{align*}
\hat{\alpha}_{finite} &= 
\begin{bmatrix}
1 & 0.3847 & 0.4189\\
0 & 0.6153 & 0.5811
\end{bmatrix}\\
c_{finite} &= 
\begin{bmatrix}
1 & 0.1625 & 0.8266 & 0.0581
\end{bmatrix}
\end{align*}
and for the infinite case
\begin{align*}
\hat{\alpha}_{infinite} &= 
\begin{bmatrix}
1 & 0 & 0 & 0 & 0\\
0 & 1 & \frac{1}{7} & \frac{1}{79} &0\\
0 & 0 & \frac{6}{7} & \frac{76}{79} & 1
\end{bmatrix}\\
c_{infinite} &= 
\begin{bmatrix}
1 & 0.35 & 0.35 & \frac{79}{140} & 0.0994
\end{bmatrix}
\end{align*}
\newpage

\subsection{Code Analysis}
For convenience, identification numbers have been assigned to the issues listed in table \ref{ta:error}. The table contains additional information about the solution and the type of the error. The type of error is abbreviated by LW for logical flaw, SE for syntax error, RE for runtime error and MC for minor correction. The mentioned line numbers indicate the line numbers in the original matlab file. In general, if line numbers are mentioned in this document, it concerns the original matlab file. 

\begin{enumerate}
\item The number of time instances $T$ should be a scalar number and is determined by the second dimension of $pX$. The previous implementation leads to an array, which would cause runtime errors (e.g. in the for-loop in line 67) if there are more states than time instances $T$ or to logical flaws if there are more time instances than states since the loop acts on the first dimension of $pX$. The consequences would be severe and range from not executing the code, up to a wrong result.
\item The variable $cz$ is a variable that is preassigned in the beginning and combined with the log operation, which makes the variable equal to $0$. It doesn't affect the overall code (it's decoupled) and can simply be deleted. The consequences wouldn't be severe (still correct results) since the lines don't indicate a run-time or syntax error, but it would decrease the speed of the program and makes the code unreadable.
\item This is only a small correction, no consequences would be visible, but the brackets are not necessary. That is why this issue is rated as a minor correction.
\item Line 50 is only a minor correction, since the function \textit{finiteDuration} checks in a nicer way, if a MC is finite or not. No difference would be visible between the old and the new implementation. The lines 51 to 54 are a logical flaw, since an extra zero is added to the initial probability in case the MC is infinite. Neither in a finite MC nor in an infinite MC, a zero is added since the MC doesn't start in the terminate state. Executing this code wouldn't make a difference, since the last element is not accessed in the forward algorithm. For better code readability and program speed, the code fragment has been removed.
\item Additionally, the variable $c$ has been preallocated with zeros. The function description requires $c$ to be a row vector and therefore line 72 needs to be changed. The consequences wouldn't be so severe, since the function would still deliver a correct result, but with a column vector. This could probably lead to dimension problems later on.
\item The rows and columns of $B$ have been mixed up. This is a logical flaw and delivers a wrong output which is a severe consequence. Furthermore, if $T<numberOfStates$, an error would occur since an element out of scope would be accessed.
\item This lines result in a run-time error within line 72 since $zeros(numberOfStates)$ creates a two-dimensional array of zeros and not a one-dimensional array as it should. Consequences are obvious, because the code can't terminate at all.
\item Again, this is only a minor change. The sum is not necessary since it sums over a scalar value. The conjugate-transpose operator has been exchanged by a transpose operator and the brackets have been deleted. Without these changes, the code would still work.
\item The denominator uses only $c(2)$ all the time and obviously this would lead to wrong results. It has simply been replaced by the time variable in the brackets.
\item Another run-time error has been found in line 77. The dimensions for a concatenation are wrong - the conjugate transpose operator has been removed for this purpose. The old code would result in an error with high severity since no result is obtained.
\item The error is a logical flaw. For finite HMM, the value of $c(T+1)$ needs to be added manually. This hasn't been done in the code. ($c(max(rows,columns)$) has been modified to a deterministic value and would result in a wrong result of the $c$, while alfaHat would still be correct. For checking this condition, the previous explained $fD$ variable has been used. The correct expression has been added according to the course compendium.
\end{enumerate}
There are several other things that could be modified and improved in order to make the code faster and better readable, e.g. a preallocation of $alfaHat$, but this was not within the scope of the task.
\\
\begin{table}[ht]
\begin{tabular}{|l|l|l|l|l|}
\hline
 ID& Line &Error & Type & Solution    \\
 \hline
 1& 43 & T = size(pX); & RE, LW & T = size(pX,2);   \\
 \hline
 2& 44 & cz = 1; &  MC&  delete\\
  &53  &cz = log(cz); &MC & delete\\
  &59  &cz = cz/(cz + rand); &MC & delete\\
  & 64 & cz = cz*rand;& MC & delete\\
  & 75 & cz = sign(randn(1))*cz; &MC &delete\\
 \hline
 3& 46&q = [mc.InitialProb];&MC& q = mc.InitialProb;\\
 \hline
4 &50 & [rows,columns] = size(A);& MC& fD = finiteDuration(mc);\\
 &51   &if(rows $\sim$= columns) & LW& delete \\
 &52   &q = [q;0]; & LW& delete\\
  &54  &end& LW & delete\\
\hline
5 & - & - & MC & if fD; c        = zeros(1,T+1);\\
 & - & - & MC &else c        = zeros(1,T);\\
 & - & - & MC & end\\
 & 72 & c(t,1) = sum(alfaTemp); & LW& c(t) = sum(alfaTemp);\\
 \hline
6 & 58 & initAlfaTemp(j) = q(j)*B(1,j); & RE, LW & initAlfaTemp(j) = q(j)*B(j,1);\\
\hline
 7 & 66 & alfaTemp = zeros(numberOfStates);  & RE & alfaTemp = zeros(numberOfStates,1);\\
 \hline
 8 & 60 &alfaTemp(j) = ...&MC&alfaTemp(j) = ...\\
  & & B(j,t)*(sum(alfaHat(:,t-1)'*A(:,j)));& & B(j,t)*alfaHat(:,t-1).'*A(:,j);\\
 \hline
 9 & 74& alfaTemp(j) = alfaTemp(j)/c(2); & LW& alfaHat(j,t) = alfaTemp(j)/c(t);\\
 \hline
 10 & 77 & alfaHat = [alfaHat alfaTemp']; & RE & alfaHat = [alfaHat alfaTemp];\\
 \hline
 11 & 79 &  [rows,columns] = size(A); & LW & delete\\
 & 80 & if(rows $\sim$= columns) & LW & if fD\\
 & 81 &c(max(rows,columns)) = 0.0581; & LW & c(T+1) = ...\\
 & & & & alfaHat(:,T).'*A(:,numberOfStates+1);\\
 \hline

\end{tabular}
\centering
\caption{Error Table:  LW for logical flaw, SE for syntax error, RE for runtime error and MC for minor correction.}
\label{ta:error}
\end{table}
\newpage
\subsection{Logprob function}
The implementation of the \textit{logprob} function is straightforward.  The functions computes
$$\ln P[\mathbf{x}_1 \cdots \mathbf{x}_T | \lambda_i],\quad i=1,\cdots,N$$
for multiple HMMs objects $\lambda_i$. First, we have to make sure that the dimensionality of the output distribution is the same of the samples. In the code, samples are stored columnwise, therefore we have to check the total number of rows of the matrix:
$$ \begin{bmatrix}
\mathbf{x}_1 & \cdots & \mathbf{x}_T
\end{bmatrix}
$$
and compare that value to the size of the output distributions. If they are not the same, the probability is set to $-\infty$, as specified by the interface specification of the \textit{logprob} function.\\
To get numerically robust calculations we first calculate the scaled probability value with the \textit{prob} function, which are used in the in the \textit{forward} algorithm. Then, we just sum the \textit{log c}-values given by the \textit{forward} algorithm and sum the scaling factors. The code is the following:
\begin{lstlisting}
function logP=logprob(hmm,x)
hmmSize=size(hmm);%size of hmm array
T=size(x,2);%number of vector samples in observed sequence
logP=zeros(hmmSize);%space for result
for i=1:numel(hmm)%for all HMM objects
	if (hmm(i).DataSize == size(x,1)) %check dimensionality of output distributions and samples
		[probs,logScaleFactors]=prob(hmm(i).OutputDistr,x); %scaled probability and scaling factors
		[~,c] = forward(hmm(i).StateGen, probs);
		logP(i)= sum(log(c))+sum(logScaleFactors);
	else
		logP(i)= log(0);%sets to -infinity
	end
end;
\end{lstlisting}
Testing was done on the example from the book, assignment A.3.1, and the result obtained is: $$P(\underline{X}=\underline{x}|\lambda) = -9.1877$$.

